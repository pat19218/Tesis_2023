%PARAFRASEAR LOS OBJETIVOS GENERAL Y ESPECÍFICOS. (Es decir, cuál es el problema o situación que se desea resolver)

%PODRÍAN MENCIONAR LO QUE SE PRESENTARÁ EN ESTE DOCUMENTO, ES DECIR, CÓMO ESTÁ ORGANIZADO EL MISMO.

En este trabajo de investigación, se aborda el tema de la epilepsia, un trastorno neurológico común caracterizado por convulsiones recurrentes e incapacitantes, conocidas como crisis epilépticas, médicamente descrito como actividad ictal. Se ha observado un creciente interés en la aplicación del aprendizaje automático en aplicaciones clínicas y experimentales relacionadas con el diagnóstico de la epilepsia y otros trastornos neurológicos y psiquiátricos. Los métodos de aprendizaje automático son considerados como potenciales herramientas capaces de proporcionar un rendimiento fiable y óptimo en el diagnóstico clínico, la predicción y la medicina personalizada, mediante la aplicación de algoritmos matemáticos y enfoques computacionales.

Entre las aplicaciones que se han empleado con los algoritmos de aprendizaje automático en el contexto de la epilepsia, se encuentra el reconocimiento de patrones de interés en señales bioeléctricas. Este enfoque ha demostrado ser efectivo en la reducción del tiempo de diagnóstico, en contraposición al método tradicional de análisis y anotaciones manuales realizadas por los médicos, que tiende a ser una tarea que consume mucho tiempo, especialmente cuando se tratan registros de varias horas de duración.

El objetivo de este proyecto de investigación es aplicar los algoritmos de aprendizaje automático desarrollados en fases anteriores a una mayor cantidad de señales bioeléctricas, y mejorar el proceso de detección de segmentos de interés en las señales, para el estudio de la epilepsia.
Esto implica el análisis de diversas señales bioeléctricas y la evaluación de clasificadores. En trabajos previos, se emplearon características en el dominio del tiempo, frecuencia y se utilizaron transformadas Wavelet para el análisis. 
En este trabajo, se presenta una revisión de las bases teóricas utilizadas para abordar el problema, así como una descripción detallada de los experimentos realizados, la metodología empleada y los resultados obtenidos en el transcurso de esta investigación. Asimismo, se presentan las conclusiones y recomendaciones que surgen de este estudio, con miras a futuras investigaciones en esta área de estudio.

