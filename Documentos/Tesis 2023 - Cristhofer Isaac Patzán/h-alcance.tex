%Podemos usar \Gls{latex} para escribir de forma ordenada una \gls{formula} matemática. 

%¿HASTA DÓNDE SE LLEGÓ CON ESTE PROYECTO? ¿QUÉ RESOLVIÓ, QUÉ QUEDÓ PARA TRABAJOS FUTUROS?  Basarse en los objetivos cumplidos.

%NO ENTRAR EN DETALLES TÉCNICOS O VALORES ESPECÍFICOS EN ESTA SECCIÓN. LA EXPLICACIÓN DE LOS ALCANCES DEBE SER MÁS GENERAL.

%Revisar los tiempos verbales. No usar tiempo futuro para describir cosas que ya hicieron.

%No incluir recomendaciones en este capítulo. Dejarlas para el capítulo de recomendaciones.

Los resultados derivados de este estudio se aplicarán principalmente en el contexto de Guatemala, aunque las metodologías desarrolladas pueden ser de utilidad en otros entornos médicos similares. Para este trabajo de graduación se recopilaron señales electroencefalográficas y electromiográficas capturadas con el equipo de la Universidad del Valle de Guatemala (UVG). Se obtuvieron señales señales bioeléctricas de pacientes con epilepsia atendidos en el Instituto HUMANA, ubicado en Guatemala. 

Los algoritmos de aprendizaje automático se aplicaron específicamente para la extracción de características de las señales bioeléctricas, la detección de patrones de interés y la generación de anotaciones relevantes. Estos algoritmos se basaron en técnicas de aprendizaje supervisado y no supervisado. No se abordó la investigación de nuevos algoritmos de aprendizaje automático en este trabajo, sino la adaptación y optimización de técnicas existentes.

La actualización de la herramienta de software propuesta se enfocó en la mejora de los algoritmos empleados, así como aplicar buenas practicas de programación, obteniendo una optimización en cuanto al costo computacional. Se incluyó la exportación de datos de los segmentos de interés a analizar. 
La implementación de tratamientos médicos y decisiones clínicas a partir de los resultados obtenidos con la herramienta, estuvo fuera del alcance de este trabajo.

 %Esta herramienta permitió la detección y anotación de eventos epilépticos, así como la generación de informes que fueron revisados por especialistas médicos para su validación.