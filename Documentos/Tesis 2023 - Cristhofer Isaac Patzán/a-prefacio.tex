%INCLUIR, EN FORMA RESUMIDA, LOS PRINCIPALES RESULTADOS OBTENIDOS.

%No hacer referencias a figuras en el resumen.

%Aquí no vale la pena mencionar lo que NO se hizo.

En Guatemala no se cuenta con fácil acceso a nuevas tecnologías para el estudio a fondo de los casos de pacientes con epilepsia, que a la fecha se estiman que son alrededor de 325,000 pacientes; una enfermedad que repercute tanto en la salud del paciente como en su desenvolvimiento social. 
El presente trabajo tiene como objetivo, aplicar los algoritmos de aprendizaje automático desarrollados en fases anteriores a una mayor cantidad de señales bioeléctricas, y mejorar el proceso de detección de segmentos de interés en las señales, para el estudio de la epilepsia.

Para lograrlo se inició con la obtención de señales bioeléctricas con el equipo BIOPAC de la Universidad del Valle de Guatemala (UVG), de personas que no sufren de ataques epilépticos y por parte de HUMANA, señales bioeléctricas de pacientes con ataques de epilepsia.
Recolectar una mayor cantidad de datos que en años previos fue de gran necesidad, sirvió para una mejor clasificación al momento de entrenar el modelo de aprendizaje automático. Cabe destacar que a los datos obtenidos con el equipo de UVG, se les asignaba un nombre según la norma que se estableció en UVG. Posteriormente se extrajo características en el dominio del tiempo, frecuencia y wavelets de las señales anteriormente mencionadas. Los experimentos para la validación de modelos, en el caso de las señales de electromiografía (EMG) fueron intrasujeto, mientras que, en el caso de las señales de electroencefalografía (EEG) fueron intersujeto. 

Por ultimo, se actualizó la herramienta de software para el estudio de la epilepsia que se ha desarrollado en los últimos años. Esto incluía el mejoramiento del proceso de detección de segmentos de interés en las señales bioeléctricas, la optimización funciones para la extracción de características y el entrenamiento del modelo de aprendizaje automático y la generación automática de anotaciones relevantes. Cumplidas estas actividades se procedió con el análisis estadístico  para evaluar el rendimiento de los algoritmos e identificar posibles mejoras a los mismos.