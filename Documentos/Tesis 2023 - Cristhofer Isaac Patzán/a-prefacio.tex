%Agradecimientos e inspiración del tema de investigación

En el transcurso de esta ardua travesía académica que culmina con la presentación de esta tesis, es necesario detenerse un momento para expresar mi sincero agradecimiento a todas las personas que han sido pilares fundamentales en este camino lleno de desafíos y logros.

En primer lugar, quiero dedicar un profundo agradecimiento a la Universidad del Valle de Guatemala y SEGEPLAN, quienes me brindaron la oportunidad de realizar mis estudios de grado con una beca completa. Esta institución no solo me proporcionó una educación de alta calidad, sino que también me apoyó económicamente a lo largo de toda mi carrera. 
Quiero dedicar un profundo agradecimiento al Dr. Luis Rivera, quien desempeñó un papel esencial como mi asesor de tesis. Su apoyo incondicional, orientación experta y dedicación a este proyecto fueron invaluables. El Dr. Rivera no solo compartió su vasto conocimiento, sino que también brindó una fuente inagotable de inspiración y motivación. Su guía y compromiso fueron esenciales para llevar a cabo esta investigación de manera exitosa.

A mi familia, en especial a mi madre, tías y abuela, les debo un reconocimiento especial. Su apoyo constante, aliento inquebrantable y amor infinito fueron los cimientos sobre los cuales construí mi camino académico. Siempre estuvieron ahí para impulsarme en los momentos difíciles y celebrar conmigo en los triunfos. Su confianza en mí fue el motor que me llevó a perseverar y alcanzar este logro.

Esta tesis es el resultado de un esfuerzo colectivo y el apoyo de muchas personas que creyeron en mí a lo largo de los años. Cada página escrita y cada descubrimiento logrado son un testimonio de la dedicación y el trabajo arduo de todos aquellos que contribuyeron de alguna manera a este proyecto.

A todos ellos, mi más profundo agradecimiento.