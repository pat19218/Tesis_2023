\newglossaryentry{latex}
{
    name=latex,
    description={Es un lenguaje de marcado adecuado especialmente para la creación de documentos científicos}
} 
\newglossaryentry{neurociencia}
{
    name=neurociencia,
    description={La neurociencia es un campo multidisciplinario que se dedica al estudio del sistema nervioso, que incluye el cerebro, la médula espinal y los nervios}
}  
\newglossaryentry{kernel}
{
    name=kernel,
    description={En el contexto de aprendizaje automático y estadísticas, el término ``kernel''se refiere a una función matemática que mide la similitud entre pares de datos en un espacio de características} 
}
\newglossaryentry{algoritmo}
{
	name=algoritmo,
	description={Un algoritmo es una secuencia finita de instrucciones o reglas bien definidas, diseñadas para realizar una tarea o resolver un problema específico} 
}

\newglossaryentry{Struct}
{
	name=Struct,
	description={Es una abreviatura de ``structura'' y se refiere a un tipo de datos que se utiliza para almacenar datos relacionados de manera organizada. Una estructura en MATLAB es un contenedor que puede contener diferentes tipos de datos, incluidos escalares, matrices, cadenas de texto y otras estructuras} 
}
\newglossaryentry{Frecuencia de muestreo}
{
	name=Frecuencia de muestreo,
	description={Se refiere al número de muestras de una señal analógica que se toman en un período de tiempo determinado para convertirla en una señal digital. En otras palabras, es la cantidad de veces que se registran o se toman mediciones de una señal analógica en un segundo} 
}

\newglossaryentry{intersujeto}
{
	name=intersujeto,
	description={Es un término que se utiliza en investigaciones y análisis para referirse a las comparaciones o análisis que se realizan entre diferentes sujetos o participantes en un estudio. Se refiere a la variabilidad o diferencias que existen entre individuos distintos en una muestra o grupo de estudio} 
}
\newglossaryentry{intrasujeto}
{
	name=intrasujeto,
	description={Se refiere a algo que ocurre o se aplica dentro del mismo sujeto o individuo. En el contexto de la investigación científica, especialmente en estudios clínicos o experimentales, el término se utiliza para describir la variabilidad o los efectos dentro de un mismo individuo a lo largo del tiempo o en diferentes condiciones} 
}

\newglossaryentry{wavelets}
{
	name=wavelets,
	description={Son funciones matemáticas que tienen una duración limitada y están localizadas tanto en el tiempo como en la frecuencia. Son utilizadas en el análisis de señales y procesamiento de imágenes para representar y analizar información de manera eficiente} 
}
\newglossaryentry{matrices de confusión}
{
	name=matrices de confusión,
	description={Es una herramienta que se utiliza en el campo de la clasificación en aprendizaje automático para evaluar el rendimiento de un modelo predictivo. Esta matriz presenta de manera sistemática la comparación entre las predicciones de un modelo y las clases reales de un conjunto de datos} 
}

\newglossaryentry{RNA}
{
	name=RNA,
	description={En el ámbito del aprendizaje automático y la inteligencia artificial, una Red Neuronal Artificial se refiere a un modelo computacional inspirado en la estructura y funcionamiento del cerebro humano} 
}

\newglossaryentry{SVM}
{
	name=SVM,
	description={Es un algoritmo de aprendizaje supervisado utilizado tanto para tareas de clasificación como de regresión. SVM es particularmente eficaz en espacios de alta dimensión y es ampliamente utilizado en problemas de aprendizaje automático y minería de datos} 
}
