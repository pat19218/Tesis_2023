En Guatemala, hay un acceso limitado a la atención médica, lo que se agrava en las zonas rurales donde la incidencia de la epilepsia es mayor. A pesar de la alta prevalencia de la epilepsia en el país, hay una escasa cantidad de investigaciones sobre esta afección, lo que hace necesario llenar este vacío en el conocimiento y obtener información importante sobre la epilepsia en esta población \cite{mendizabal1996prevalence}.

Para la continuación de esta línea de investigación se percibe la necesidad de ampliar la herramienta incorporando señales bioeléctricas de interés, como las de electromiograma (EMG). Asimismo, se ha recomendado continuar experimentando el rendimiento de los clasificadores con diferentes técnicas de aprendizaje automático no supervisado para señales EEG, ECG y otras señales bioeléctricas relacionadas con la epilepsia. Para la investigación que se realizo en el presente proyecto, se aplico la mejora del proceso de selección de características de las señales mediante asesoría médica y la validación constante con especialistas en el campo para mejorar la predicción de los clasificadores \cite{camila_2022}. 

El acceso a un mayor número de señales proporciona una muestra más representativa de la actividad bioeléctrica cerebral y muscular de los pacientes epilépticos. Esto permite una comprensión más completa y precisa de los patrones y características de las señales relacionadas con la epilepsia, lo que conduce a una mejor comprensión de la enfermedad y sus diversas manifestaciones \cite{rouhiainen2018inteligencia}.

La aplicación de algoritmos de agrupación a un mayor número de señales de EEG y EMG puede ayudar a identificar subgrupos o patrones ocultos dentro de la población de pacientes con epilepsia \cite{felix2002data}. Estos subgrupos pueden tener relevancia clínica, como diferentes tipos de epilepsia, respuestas variadas al tratamiento o diferencias en la gravedad de los síntomas. Al identificar estos subgrupos, los profesionales sanitarios pueden adaptar el tratamiento y los cuidados de forma más precisa a cada paciente, personalizando así su enfoque médico.

En este trabajo se busca, además de, aplicar algoritmos de aprendizaje automático a una mayor cantidad de señales bioeléctricas también, implementar un algoritmo que pueda indicar en qué tiempo de la grabación se encuentran señales de interés para su estudio, con el fin de aprovechar de forma óptima el tiempo disponible. Actualmente, las grabaciones pueden durar hasta 24 horas y los médicos deben analizar toda la grabación para detectar la actividad bioeléctrica de interés. Este proceso resulta muy ineficiente.

