In Guatemala there is no easy access to new technologies for the in-depth study of the cases of patients with epilepsy, which to date are estimated to be around 325,000 patients. A disease that affects both the patient's health and social development. 
The present work aims to apply the machine learning algorithms developed in previous phases to a larger number of bioelectrical signals, and to improve the process of detecting segments of interest in the signals, for the study of epilepsy.

To achieve this, we began by obtaining bioelectrical signals with the BIOPAC equipment of the Universidad del Valle de Guatemala (UVG), from people who do not suffer from epileptic seizures, and by HUMANA, bioelectrical signals from patients with epileptic seizures.
Collecting a larger amount of data, which in previous years was of great need, served for a better classification when training the machine learning model. It should be noted that the data obtained with the UVG equipment were assigned a name according to the standard that was established in UVG. Subsequently, time, frequency and wavelet domain features were extracted from the aforementioned signals. The experiments for model validation, in the case of electromyography (EMG) signals, were intrasubject, while in the case of electroencephalography (EEG) signals they were intersubject. 

Finally, the software tool for the study of epilepsy that has been developed in recent years was updated. This included improving the process of detecting segments of interest in the bioelectrical signals, optimizing functions for feature extraction and training of the machine learning model and automatic generation of relevant annotations. Once these activities were completed, statistical analysis was performed to evaluate the performance of the algorithms and identify possible improvements to them.